\theoremstyle{remark}
\newtheorem{rem}{Remark}

\section{Game rules}

\begin{enumerate}
\item There are the following pieces:
  \begin{itemize}
  \item Pawn \pawn
  \item Knight \knight
  \item Bishop \bishop
  \item Rook \rook
  \item Queen \queen
  \item King \king
  \end{itemize}
\item The starting position is as in Figure~\ref{fig:reset}.
  \begin{figure}[!ht]
    \centering
    \newgame\showboard
    \caption{The starting position}
    \label{fig:reset}
  \end{figure}
\item Players move in turns, one move each, starting with the White,
  until the game ends.
\item The ending condition for a game is one of:
  \begin{enumerate}
  \item \emph{Draw by Stalemate}
  \item \emph{Draw by threefold repeat} 
  \item \emph{Checkmate}
  \item \emph{Resign}
  \end{enumerate}
\end{enumerate}

\section{Implementation}

\begin{itemize}
\item Custom type \verb+gamemove+, to represent a move. Includes:
  \begin{itemize}
  \item \verb+dscore :real+, as in ``difference in score''
  \item \verb+mine :d_chess_square+, specifying $(x_1,y_1)\mapsto(x_2,y_2)$.
  \end{itemize}

\item Custom type \verb+gamestate+, to represent the state of a
  game. Includes:
  \begin{itemize}
  \item \verb+score :real+
  \item \verb+moves :gamemove[]+, previous moves
  \item \verb+next :gamemove[]+, next possible moves
  \end{itemize}

  \begin{rem} Next possible moves are cached for efficiency, because
    the same set of moves is needed more than once:
    \begin{itemize}
    \item a move $M$ is valid only if the moves immediately following
      $M$ cannot ``capture the King''
    \item after having applied $M$, the computer player needs to
      consider all the moves following $M$
    \end{itemize}
  \end{rem}

\item \emph{prevalid} move: move allowed by the movement rules of the
  piece

\item \emph{valid} move: prevalid move $M$\quad +\quad the moves
  immediately following $M$ cannot ``capture the King''

\item Temporary table \verb+my_states+ represents a set of objects of
  type \verb+gamestate+.

\item Temporary table \verb+my_moves+ represents a set of objects of
  type \verb+gamemove+.

\item The side of the player is represented by a \verb+boolean+ (Black
  and White respectively \verb+f+ and \verb+t+)

\item Function \verb+prevalid_moves(gamestate,boolean)+ computes the
  set of prevalid moves, starting from a given state, and assuming
  that a given side is going to play next.
  \begin{rem}
    In a normal chess game the side who is playing next depends on the
    length of the \verb+G.moves[]+ array (Black if odd, White if
    even). We allow to specify it explicitely to reduce the frequency
    of that computation, and also to allow non-standard games, such as
    chess problems.
  \end{rem}


\section{Optimisation}



\end{itemize}

%%% Local Variables: 
%%% mode: latex
%%% TeX-master: "main-screen"
%%% End: 
